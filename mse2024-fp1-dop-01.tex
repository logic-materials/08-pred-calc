%! suppress = Makeatletter
%! suppress = TooLargeSection
%! suppress = MissingLabel
\documentclass{article}

% Fields
\usepackage{geometry}
\geometry{top=25mm}
\geometry{bottom=35mm}
\geometry{left=20mm}
\geometry{right=20mm}
% ------------------------------------------------

% Graphics
\usepackage{color}
\usepackage{tabularx}
\usepackage{tikz}
\usepackage{blkarray}
\usepackage{graphicx}
% ------------------------------------------------

% Math
\usepackage{amsmath, amsfonts}
\usepackage{amssymb}
\usepackage{proof}
\usepackage{mathrsfs}
% Crossed-out symbols
% https://tex.stackexchange.com/questions/75525/how-to-write-crossed-out-math-in-latex
\usepackage[makeroom]{cancel}
\usepackage{mathtools}
% ------------------------------------------------

% Additional font sizes
% https://www.overleaf.com/learn/latex/Questions/How_do_I_adjust_the_font_size%3F
\usepackage{moresize}
% Additional colors
% https://www.overleaf.com/learn/latex/Using_colours_in_LaTeX
\usepackage{xcolor}
% \texttimes
\usepackage{textcomp}
% ------------------------------------------------

% Language
\usepackage[utf8] {inputenc}
\usepackage[T2A] {fontenc}
\usepackage[english, russian] {babel}
\usepackage{indentfirst, verbatim}
\usetikzlibrary{cd, babel}
% ------------------------------------------------

% Fonts
\usepackage{stmaryrd}
\usepackage{cmbright}
\usepackage{wasysym}
% ------------------------------------------------

% Code
% https://tex.stackexchange.com/questions/99475/how-to-invoke-latex-with-the-shell-escape-flag-in-texstudio-former-texmakerx
% Colored, requires --shell-escape compiling option
% \usepackage{minted}
% \setminted{xleftmargin=\parindent, autogobble, escapeinside=\#\#}
\usepackage{listings}
% ------------------------------------------------

% Custom envs
% https://tex.stackexchange.com/questions/371286/draw-a-horizontal-line-in-latex
\newenvironment{proof}{\subparagraph{\hspace{-1em}Решение:\newline}}{\par\noindent\rule{\textwidth}{0.4pt}}
% ------------------------------------------------

% Custom commands
\newcommand{\comb}[1]{\mathbf{#1}}
\newcommand{\step}{\rightsquigarrow}
\newcommand{\term}[1]{\mathbf{#1}}
\newcommand{\ap}{~}
\newcommand{\termdef}{\coloneqq}
\newcommand{\subst}[3]{\left[#2 \mapsto #3 \right] #1}
\newcommand{\eqbeta}{=_\beta}
\newcommand{\eqeta}{=_\eta}
\newcommand{\tlist}[1]{\term{[}#1\term{]}} % list-term
\renewcommand{\emph}[1]{{\color{blue} \textit{#1}}}
% ------------------------------------------------

% Head
\usepackage{fancybox,fancyhdr}
\usepackage{hyperref}
\pagestyle{fancy}
\fancyhead[R]{Студент(ка) Студентов(а)} % TODO введите ваше имя
\fancyhead[L]{ИТМО MSE, ФП 2024, Дз 1}
% ------------------------------------------------

% Numbering
% https://tex.stackexchange.com/questions/80113/hide-section-numbers-but-keep-numbering
\makeatletter
\renewcommand\thesubsection{Блок \@arabic\c@subsection.\hspace{-0.8em}}
\renewcommand\thesubsubsection{Задание \@arabic\c@subsubsection\hspace{-0.8em}}
% https://tex.stackexchange.com/questions/327689/numbering-subsubsections-with-letters
\renewcommand\theparagraph{\alph{paragraph})\hspace{-0.8em}}
% https://tex.stackexchange.com/questions/129208/numbering-paragraphs-in-latex
\setcounter{secnumdepth}{4}
\makeatother
% ------------------------------------------------

\begin{document}

    \section*{Дополнительные задания 1. Лямбда-исчисление}

    \subsubsection{(1б)}

    Задайте лямбда-терм $\mathbf{xor}~b_1~b_2$.

    \begin{proof}
        TODO % TODO
    \end{proof}

    \subsubsection{(1б)}

    Задайте терм, который принимает на себя бинарную функцию $\oplus$ над
    логическими переменными и возвращает её таблицу истинности: пару из
    пары, в которой содержатся $\mathbf{true} \oplus \mathbf{true}$ и
    $\mathbf{true} \oplus \mathbf{false}$, и пары, в которой содержатся
    $\mathbf{false} \oplus \mathbf{true}$ и $\mathbf{false} \oplus
    \mathbf{false}$.

    \begin{proof}
        TODO % TODO
    \end{proof}

    \subsubsection{(1.5б)}

    Задайте терм, который принимает на вход таблицу истинности функции
    $\oplus$ над логическими переменными (представление таблицы описано в
    предыдущем задании) и возвращает терм, вычисляющий $\oplus$.

    \begin{proof}
        TODO % TODO
    \end{proof}

    \subsubsection{(1б)}

    Докажите, что $\mathbf{mult}~a~(\mathbf{plus}~b~c) =
    \mathbf{plus}~(\mathbf{mult}~a~b)~(\mathbf{mult}~a~c)$, если $a$, $b$ и $c$ "---
    числа Чёрча.

    \begin{proof}
        TODO % TODO
    \end{proof}

    \subsubsection{(3б)}

    Не используя комбинатор неподвижной точки, задайте терм, который
    принимает на вход два числа Чёрча и возвращает наибольшее из них.

    \begin{proof}
        TODO % TODO
    \end{proof}

    \subsubsection{(5б)}

    Задайте терм $\mathbf{mult}$, используя только аппликацию, скобки и
    термы $\mathbf{S}$ и $\mathbf{K}$. Использовать лямбда-абстракцию
    запрещается.

    \begin{proof}
        TODO % TODO
    \end{proof}

\end{document}
