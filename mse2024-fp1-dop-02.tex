%! suppress = Makeatletter
%! suppress = TooLargeSection
%! suppress = MissingLabel
\documentclass{article}

% Fields
\usepackage{geometry}
\geometry{top=25mm}
\geometry{bottom=35mm}
\geometry{left=20mm}
\geometry{right=20mm}
% ------------------------------------------------

% Graphics
\usepackage{color}
\usepackage{tabularx}
\usepackage{tikz}
\usepackage{blkarray}
\usepackage{graphicx}
% ------------------------------------------------

% Math
\usepackage{amsmath, amsfonts}
\usepackage{amssymb}
\usepackage{proof}
\usepackage{mathrsfs}
% Crossed-out symbols
% https://tex.stackexchange.com/questions/75525/how-to-write-crossed-out-math-in-latex
\usepackage[makeroom]{cancel}
\usepackage{mathtools}
% ------------------------------------------------

% Additional font sizes
% https://www.overleaf.com/learn/latex/Questions/How_do_I_adjust_the_font_size%3F
\usepackage{moresize}
% Additional colors
% https://www.overleaf.com/learn/latex/Using_colours_in_LaTeX
\usepackage{xcolor}
% \texttimes
\usepackage{textcomp}
% ------------------------------------------------

% Language
\usepackage[utf8] {inputenc}
\usepackage[T2A] {fontenc}
\usepackage[english, russian] {babel}
\usepackage{indentfirst, verbatim}
\usetikzlibrary{cd, babel}
% ------------------------------------------------

% Fonts
\usepackage{stmaryrd}
\usepackage{cmbright}
\usepackage{wasysym}
% ------------------------------------------------

% Code
% https://tex.stackexchange.com/questions/99475/how-to-invoke-latex-with-the-shell-escape-flag-in-texstudio-former-texmakerx
% Colored, requires --shell-escape compiling option
% \usepackage{minted}
% \setminted{xleftmargin=\parindent, autogobble, escapeinside=\#\#}
\usepackage{listings}
% ------------------------------------------------

% Custom envs
% https://tex.stackexchange.com/questions/371286/draw-a-horizontal-line-in-latex
\newenvironment{proof}{\subparagraph{\hspace{-1em}Решение:\newline}}{\par\noindent\rule{\textwidth}{0.4pt}}
% ------------------------------------------------

% Custom commands
\newcommand{\comb}[1]{\mathbf{#1}}
\newcommand{\step}{\rightsquigarrow}
\newcommand{\term}[1]{\mathbf{#1}}
\newcommand{\ap}{~}
\newcommand{\termdef}{\coloneqq}
\newcommand{\subst}[3]{\left[#2 \mapsto #3 \right] #1}
\newcommand{\eqbeta}{=_\beta}
\newcommand{\eqeta}{=_\eta}
\newcommand{\tlist}[1]{\term{[}#1\term{]}} % list-term
\renewcommand{\emph}[1]{{\color{blue} \textit{#1}}}
% ------------------------------------------------

% Head
\usepackage{fancybox,fancyhdr}
\usepackage{hyperref}
\pagestyle{fancy}
\fancyhead[R]{Студент(ка) Студентов(а)} % TODO введите ваше имя
\fancyhead[L]{ИТМО MSE, ФП 2024, Дз 1}
% ------------------------------------------------

% Numbering
% https://tex.stackexchange.com/questions/80113/hide-section-numbers-but-keep-numbering
\makeatletter
\renewcommand\thesubsection{Блок \@arabic\c@subsection.\hspace{-0.8em}}
\renewcommand\thesubsubsection{Задание \@arabic\c@subsection.\@arabic\c@subsubsection\hspace{-0.8em}}
% https://tex.stackexchange.com/questions/327689/numbering-subsubsections-with-letters
\renewcommand\theparagraph{\alph{paragraph})\hspace{-0.8em}}
% https://tex.stackexchange.com/questions/129208/numbering-paragraphs-in-latex
\setcounter{secnumdepth}{4}
\makeatother
% ------------------------------------------------

\begin{document}

    \section*{Дополнительные задания 2. Больше лямбда-исчисления}

    \subsection{Задачи}

    \subsubsection{(1б)}

    Известно два определения увеличения числа Чёрча на единицу:
    $$\begin{array}{l}
        \mathbf{suc} := \lambda n~s~z\ldotp s~(n~s~z) \\
        \mathbf{suc'} := \lambda n~s~z\ldotp n~s~(s~z) \\
    \end{array}$$
    
    По аналогии с $\mathbf{suc}$ можно
    построить функцию $\mathbf{cons}$ "--- терм, который дописывает в начало
    списка новый элемент. Что получится по аналогии с $\mathbf{suc'}$?

    \begin{proof}
        TODO % TODO
    \end{proof}

    \subsubsection{(1.5б)}

    Задайте лямбда-терм, который принимает на вход список и сильно
    нормализуемо возвращает результат его разворота:
    $$\mathbf{reverse}~[1, 3, 6, 7] = [7, 6, 3, 1]$$

    \begin{proof}
        TODO % TODO
    \end{proof}

    \subsubsection{(1б)}

    Реализуйте функцию, которая принимает на вход $f$ и $n$ и возвращает
    $$\prod_{i = 0}^{n-1} f(i)$$

    \begin{proof}
        TODO % TODO
    \end{proof}

    \subsubsection{(1б)}

    Представьте в кодировании по Чёрчу бинарное дерево
    \begin{lstlisting}[language=Haskell]
    data Tree a = Leaf | Node (Tree a) a (Tree a)
    \end{lstlisting}

    \begin{proof}
        TODO % TODO
    \end{proof}

    \subsubsection{(1.5б)}

    Даны двоичные числа с ведущими нулями в лямбда-исчислении
    (\emph{закодированные по Чёрчу}):
    $$\begin{array}{ll}
        \mathbf{ZB}&:= \lambda~f_0~f_1~z.~z \\
        \mathbf{IB}&:= \lambda~n~f_0~f_1~z.~f_1~(n~f_0~f_1~z) \\
        \mathbf{OB}&:= \lambda~n~f_0~f_1~z.~f_0~(n~f_0~f_1~z)
    \end{array}$$
    \textbf{ZB} "--- конструктор нуля, \textbf{IB} "--- конструктор,
    принимающий $n$ и возвращающий $2 \cdot n + 1$, \textbf{OB} "---
    конструктор, принимающий $n$ и возвращающий $2 \cdot n$. Таким
    образом, последовательность \textbf{IB} и \textbf{OB} кодирует биты
    некоторого числа.

    У этих чисел есть небольшая проблема: они поддерживают несколько по смыслу
    идентичных, но структурно различающихся способов записи каждого числа,
    поскольку \textbf{ZB}, \textbf{OB ZB}, \textbf{OB (OB ZB)} и так далее все
    могут кодировать число 0.

    Несколько примеров чисел в таком кодировании:
    \begin{center}
        \begin{tabular}{|c|c|c|}
            \hline
            0 & \texttt{0} & \textbf{ZB} \\
            1 & \texttt{1} & \textbf{IB ZB} \\
            2 & \texttt{10} & \textbf{OB (IB ZB)} \\
            3 & \texttt{11} & \textbf{IB (IB ZB)} \\
            4 & \texttt{100} & \textbf{OB (OB (IB ZB)} \\
            5 & \texttt{101} & \textbf{IB (OB (IB ZB)} \\
            6 & \texttt{110} & \textbf{OB (IB (IB ZB)} \\
            003 & \texttt{0011} & \textbf{IB (IB (OB (OB ZB)))} \\
            0002 & \texttt{00010} & \textbf{OB (IB (OB (OB (OB ZB))))} \\
            \hline
        \end{tabular}
    \end{center}

    Напишите функцию, которая находит количество битов в таком числе, считая и
    ведущие нули.

    \begin{proof}
        TODO % TODO
    \end{proof}

    \subsubsection{(1.5б)}
    Напишите функцию, которая находит количество единичных битов в двоичном числе.

    \begin{proof}
        TODO % TODO
    \end{proof}

    \subsubsection{(1.5б)}
    Напишите функцию, которая находит количество битов в таком числе, не считая ведущие нули.
    
    \begin{proof}
        TODO % TODO
    \end{proof}

    \subsubsection{(1.5б)}
    Напишите функцию, которая прибавляет единицу к двоичному числу.

    \begin{proof}
        TODO % TODO
    \end{proof}
    
\end{document}
